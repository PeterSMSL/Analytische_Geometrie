\documentclass{article}
\usepackage[utf8]{inputenc}
\usepackage{amsmath}
\usepackage{amsfonts}
\usepackage{amssymb}
\usepackage{geometry}
\usepackage{enumitem}
\usepackage{framed}

\geometry{a4paper, margin=2.5cm}

\title{ANALYTISCHE GEOMETRIE – LINEARE GLEICHUNGSSYSTEME}
\author{}
\date{}

\begin{document}

\maketitle

\section{Lineare Gleichungssysteme (LGS)}

Wir können eine Gleichung mit einer Variablen (Unbekannten) lösen. Das Ganze kennen wir auch schon für mehrere Gleichungen und mehrere Variablen. Aus der Mittelstufe kennen wir vor allem die Gleichungssysteme mit zwei Gleichungen und zwei Variablen. Aus der jüngeren Vergangenheit (Rekonstruktion von Funktionen) kennen wir auch größere LGS. Zum Einstieg in die Analytische Geometrie / Lineare Algebra bietet sich eine kurze Wiederholung der Lösungsmethoden für LGS an.

\subsection{Lineare Gleichungssysteme mit zwei Gleichungen und zwei Variablen}

Hier sind das Grundverständnis der Lösungsansätze Gleichsetzungs-, Einsetzungs-, Additions- und Subtraktionsverfahren vorausgesetzt. Wir betrachten ein Beispiel zum Wiedereinstieg und dabei zwei Lösungswege:

\textbf{Beispiel 1} Löse das gegebene LGS.
\begin{align}
\text{(I)} \quad x + y &= 2\\
\text{(II)} \quad x - y &= -8
\end{align}

\textbf{a) Lösung mit Additionsverfahren}
\begin{align}
\text{I.} \quad x + y &= 2\\
\text{II.} \quad x - y &= -8\\
\hline
2x &= -6 \quad |\text{I + II}\\
x &= -3 \quad |: 2
\end{align}

$\mathbf{x = -3}$ in I einsetzen:
\begin{align}
-3 + y &= 2 \quad |+3\\
\mathbf{y} &= \mathbf{5}
\end{align}

Die Lösungsmenge des LGS beträgt somit $\mathbf{L = \{(-3|5)\}}$.

\textbf{b) Lösung mit Gleichsetzungsverfahren}
\begin{align}
\text{I.} \quad x + y &= 2 \quad |- y\\
\text{I.} \quad x &= 2 - y\\
\text{II.} \quad x - y &= -8 \quad |+ y\\
\text{II.} \quad x &= -8 + y
\end{align}

Durch Gleichsetzen von I und II folgt:
\begin{align}
2 - y &= -8 + y\\
\mathbf{y} &= \mathbf{5}
\end{align}

in I. einsetzen liefert: $\mathbf{x = -3}$ und somit ebenfalls als Lösungsmenge des LGS $\mathbf{L = \{(-3|5)\}}$.

\textbf{Beispiel 2:} Löse das gegebene LGS.
\begin{align}
\text{(I)} \quad x + y &= 9\\
\text{(II)} \quad 4x + 2y &= 24
\end{align}

\textbf{Lösung:}
\begin{align}
\text{(I)} \quad x + y &= 9\\
\text{(II)} \quad 4x + 2y &= 24 \quad |\cdot (-2)\\
\text{(III)} \quad 2x &= 6
\end{align}

Nebenrechnung: $-2x - 2y = -18$

Aus Gleichung (III) erhalten wir: $\mathbf{x = 3}$. Durch Einsetzen von $x = 3$ in (I) erhalten wir:
\begin{align}
\text{(I)} \quad 3 + y &= 9\\
\mathbf{y} &= \mathbf{6}
\end{align}

Die Lösungsmenge des LGS beträgt somit $L = \{(3|6)\}$.

\textbf{Hinweis:} Es gibt unterschiedliche Notationen. Soweit der Weg nachvollziehbar ist, ist vieles erlaubt. Meist handelt es sich bei LGS um Nebenrechnungen für ein größeres Ziel.

\newpage

\subsection{LGS mit drei Variablen – Dreieckssysteme}

Bei LGS mit mehr als zwei Gleichungen und Unbekannten werden die obigen Verfahren sehr schnell unübersichtlich und fehlerbehaftet. Deswegen geht man in diesen Fällen zu einem allgemeingültigen, immer anwendbaren Verfahren über, dem sogenannten Gauß-Algorithmus.

Der deutsche Mathematiker Carl Friedrich Gauß (1777 – 1855) hat zu Linearen Gleichungssystemen (LGS) einen Algorithmus entwickelt, der das LGS in ein Dreieckssystem überführt, mit welchem man dann schnell die gesuchte Lösung für $x$, $y$ und $z$ findet.

\textbf{Beispiel 3:}
\begin{align}
\text{(I)} \quad 3x - 2y + 4z &= 11\\
\text{(II)} \quad 4y + 2z &= 14\\
\text{(III)} \quad 5z &= 15
\end{align}

Aus Gleichung (III) ergibt sich: $\mathbf{z = 3}$.

Dieses Ergebnis setzen wir in (II) ein und lösen nach $y$ auf:
\begin{align}
4y + 2 \cdot 3 &= 14\\
\mathbf{y} &= \mathbf{2}
\end{align}

Nun setzen wir $z = 3$ und $y = 2$ in (I) ein und lösen nach $x$ auf:
\begin{align}
3x - 2 \cdot 2 + 4 \cdot 3 &= 11\\
\mathbf{x} &= \mathbf{1}
\end{align}

\textbf{Resultat:} Das gegebene Dreieckssystem ist eindeutig lösbar und besitzt die Lösungsmenge $\mathbf{L = \{(1|2|3)\}}$.

\begin{framed}
\textbf{Übung 1} Löse das Dreieckssystem und gib die Lösungsmenge an.

\textbf{a)}
\begin{align}
\text{(I)} \quad 2x + 4y - z &= -13\\
\text{(II)} \quad 2y - 2z &= -12\\
\text{(III)} \quad 3z &= 9
\end{align}

\textbf{b)}
\begin{align}
\text{(I)} \quad 2x + 4y - 3z &= 3\\
\text{(II)} \quad -6y + 5z &= 7\\
\text{(III)} \quad 2z &= 4
\end{align}
\end{framed}

\subsection{LGS mit drei Variablen – Das Gauß-Verfahren}

Ist ein LGS mit drei Variablen noch nicht in Dreiecksform, sondern zum Beispiel "voll besetzt", dann wenden wir das sogenannte Gauß-Verfahren an.

Dazu gehen wir wie folgt vor:
\begin{itemize}
\item Elimination von $x$ in Gleichung (II) und (III)
\item Elimination von $y$ in Gleichung (III)
\item Lösung des entstandenen Dreieckssystems (wie in 1.2)
\end{itemize}

\newpage

\textbf{Beispiel 4:}
\begin{align}
\text{(I)} \quad 3x + 3y + 2z &= 5\\
\text{(II)} \quad 2x + 4y + 3z &= 4\\
\text{(III)} \quad -5x + 2y + 4z &= -9
\end{align}

\textbf{1. Elimination von $x$}

Tabellarische Schreibweise:
\begin{align}
\text{(I)} \quad 3x + 3y + 2z &= 5\\
\text{(II)} \quad -6y - 5z &= -2 \quad \rightarrow 2 \cdot (I) - 3 \cdot (II)\\
\text{(III)} \quad 21y + 22z &= -2 \quad \rightarrow 3 \cdot (III) + 5 \cdot (I)
\end{align}

\textbf{2. Elimination von $y$}
\begin{align}
\text{(I)} \quad 3x + 3y + 2z &= 5\\
\text{(II)} \quad -6y - 5z &= -2\\
\text{(III)} \quad 9z &= -18 \quad \rightarrow 2 \cdot (III) + 7 \cdot (II)
\end{align}

\textbf{Dreieckssystem}

Aus Gleichung (III) ergibt sich: $\mathbf{z = -2}$.

Dieses Ergebnis setzen wir in (II) ein und lösen nach $y$ auf:
\begin{align}
-6y - 5 \cdot (-2) &= -2\\
\mathbf{y} &= \mathbf{2}
\end{align}

Nun setzen wir $z = -2$ und $y = 2$ in (I) ein und lösen nach $x$ auf:
\begin{align}
3x + 3 \cdot 2 + 2 \cdot (-2) &= 5\\
\mathbf{x} &= \mathbf{1}
\end{align}

\textbf{Resultat:} Das gegebene Dreieckssystem ist eindeutig lösbar und besitzt die Lösungsmenge $\mathbf{L = \{(1|2|-2)\}}$.

Alternativ kann die Lösungsmenge auch mit $\mathbf{(x; y; z) = (1; 2; -2)}$ notiert werden.

\newpage

\subsection{Lösbarkeit von LGS}

Für ein LGS muss es nicht immer eine eindeutige Lösung geben.

\textbf{Beispiel 5:} Untersuche das LGS mithilfe des Gauß'schen Algorithmus auf Lösbarkeit.

\begin{minipage}[t]{0.48\textwidth}
Gleichung (III) des Dreieckssystems wird als \textbf{Widerspruchszeile} bezeichnet. Sie ist unlösbar ($0x + 0y + 0z = -1$ ist für kein Tripel $x, y, z$ erfüllt).

Damit ist das Dreieckssystem als Ganzes unlösbar.

Es folgt: Das ursprüngliche LGS ist ebenfalls unlösbar, die Lösungsmenge ist daher leer: $L = \{\}$

Die Unlösbarkeit eines LGS wird nach Anwendung des Gaußschen Algorithmus stets auf diese Weise offenbar:

Wenigstens in einer Gleichung des resultierenden Dreieckssystems tritt ein offensichtlicher Widerspruch auf.
\end{minipage}%
\hfill%
\begin{minipage}[t]{0.48\textwidth}
Gleichung (III) des Gleichungssystems wird als \textbf{Nullzeile} bezeichnet. Sie ist für jedes Tripel $x, y, z$ erfüllt, stellt keine Einschränkung dar und könnte daher auch weggelassen werden.

Es verbleiben 2 Gleichungen mit 3 Variablen, von denen daher eine Variable frei wählbar ist. Wir setzen für diese "überzählige" Variable einen Parameter ein.

Wählen wir $z = c$, so folgt aus (II) $y = 2c - 1$ und dann aus (I) $x = c + 1$, mit $c \in \mathbb{R}$.

Wir erhalten für jeden Wert des freien Parameters $c$ genau ein Lösungstripel $x, y, z$. Das Gleichungssystem hat eine einparametrige unendliche Lösungsmenge: $L = \{(c + 1; 2c - 1; c); c \in \mathbb{R}\}$
\end{minipage}

\begin{framed}
\textbf{Übung 2} Untersuche das LGS auf Lösbarkeit. Gib die Lösungsmenge an.
\end{framed}

\newpage

\textbf{Zusammenfassung:}

\begin{enumerate}
\item Gauß'schen Algorithmus auf das LGS anwenden. Es entsteht eine Dreiecks- bzw. Stufenform.
\item Prüfen, welche der folgenden Eigenschaften das resultierende LGS besitzt.
\end{enumerate}

\begin{center}
\begin{tabular}{|c|c|c|}
\hline
\textbf{Widerspruch} & \multicolumn{2}{c|}{\textbf{Es existiert kein Widerspruch.}} \\
\hline
Wenigstens eine Gleichung stellt einen offensichtlichen Widerspruch dar. & Die Anzahl der Variablen ist gleich der Anzahl der nichttrivialen Zeilen. & Es gibt mehr Variable als nichttriviale Zeilen. \\
\hline
\textbf{3. Interpretation} & & \\
\hline
Das LGS ist unlösbar. & Das LGS ist eindeutig lösbar. & Das LGS hat unendlich viele Lösungen. \\
\hline
& Die einzige Lösung wird durch Rückwärtseinsetzen aus dem Stufenform-LGS bestimmt. & Die freien Parameter werden festgelegt. Die Parameterdarstellung der Lösungsmenge wird bestimmt. \\
\hline
\end{tabular}
\end{center}

\begin{framed}
\textbf{Übung 3} Untersuche das LGS auf Lösbarkeit und gib gegebenenfalls die Lösung an.
\end{framed}

\begin{framed}
\textbf{Übung 4} Eine dreistellige natürliche Zahl hat die Quersumme 16. Die Summe der ersten beiden Ziffern ist um 2 größer als die letzte Ziffer. Addiert man zum Doppelten der mittleren Ziffer die erste Ziffer, so erhält man das Doppelte der letzten Ziffer. Wie heißt die Zahl?
\end{framed}

\end{document}